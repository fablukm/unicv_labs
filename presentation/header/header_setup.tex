
%%%%%%%%%%%%%%%%%%%%%%%%%%%%%%%%%%%%%%%%%%%%%%%%%%%%%%%%%%%%%%%%%%%%%%%%%%%
% Package setup
%%%%%%%%%%%%%%%%%%%%%%%%%%%%%%%%%%%%%%%%%%%%%%%%%%%%%%%%%%%%%%%%%%%%%%%%%%%

\mathtoolsset{showonlyrefs=true}

\captionsetup{format=hang,justification=raggedright}

%%%%%%%%%%%%%%%%%%%%%%%%%%%%%%%%%%%%%%%%%%%%%%%%%%%%%%%%%%%%%%%%%%%%%%%%%%%
% Plotting
%%%%%%%%%%%%%%%%%%%%%%%%%%%%%%%%%%%%%%%%%%%%%%%%%%%%%%%%%%%%%%%%%%%%%%%%%%%

\newlength{\plotsize}

\makeatletter
\newcommand*{\overlaynumber}{\number\beamer@slideinframe}

\usetikzlibrary{external}
\usepackage{pdftexcmds}% for \pdfmdfivesum for tikz externalisation
\tikzset{
	external/system call={lualatex
		\tikzexternalcheckshellescape -halt-on-error -interaction=batchmode
		-jobname "\image" "\texsource"},
	external/only named=true,
	%%% see tex.stackexchange.com/a/119440/42225
	beamer externalizing/.style={%
		execute at end picture={%
			\tikzifexternalizing{%
				\ifbeamer@anotherslide%
				\pgfexternalstorecommand{\string\global\string\beamer@anotherslidetrue}%
				\fi%
			}{}%
		}%
	},
	external/optimize=false,
	every picture/.style={beamer externalizing}
}
\let\origtikzsetnextfilename=\tikzsetnextfilename
\renewcommand\tikzsetnextfilename[1]{\origtikzsetnextfilename{#1-\overlaynumber}}
\makeatother

\tikzexternalize[prefix=figures/ext/]

%%% for correct rendering of transparency, see tex.stackexchange.com/a/101283/42225
%%% for PDFLATEX!
%\pdfpageattr{/Group <</S /Transparency /I true /CS /DeviceRGB>>}
%%% for XELATEX!
%\usepackage{everypage}
%\AddEverypageHook{%
%	\makeatletter%
%	\special{pdf: put @thispage <</Group << /S /Transparency /I true /CS /DeviceRGB>> >>}%
%	\makeatother%
%}%

%%%%%%%%%%%%%%%%%%%%%%%%%%%%%%%%%%%%%%%%%%%%%%%%%%%%%%%%%%%%%%%%%%%%%%%%%%%
% Technical details
%%%%%%%%%%%%%%%%%%%%%%%%%%%%%%%%%%%%%%%%%%%%%%%%%%%%%%%%%%%%%%%%%%%%%%%%%%%

\newcommand{\normall}[1]{\mathopen{\displaystyle#1}}  % complements \bigl, \Bigl, \biggl, etc.
\newcommand{\normalr}[1]{\mathclose{\displaystyle#1}} % complements \bigr, \Bigr, \biggr, etc.

% No spurious spacing introduced by \left and \right, see http://tex.stackexchange.com/a/2610/42225
\let\oldleft\left
\let\oldright\right
\renewcommand{\left}{\mathopen{}\mathclose\bgroup\oldleft}
\renewcommand{\right}{\aftergroup\egroup\oldright}

% Somehow, \vec has problems for single letters, see e.g. \left|\vec{\xi}\right|, where the arrow touches the bar. This redefinition adds an empty second character to every argument, thus avoiding the problem.
\let\oldvec\vec
\def\vec#1{\oldvec{#1{}}}

% shifts in 18th of 1em: "\,": +3, "\:": +4, "\;": +5, "\!": -3
\newcommand{\?}{\:\!} % this shifts by +1/18th em

%% define every float to have \centering attribute, see http://tex.stackexchange.com/q/2651
%\makeatletter
%\g@addto@macro\@floatboxreset\centering
%\makeatother

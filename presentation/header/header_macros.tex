
%%%%%%%%%%%%%%%%%%%%%%%%%%%%%%%%%%%%%%%%%%%%%%%%%%%%%%%%%%%%%%%%%%%%%%%%%%%
% Differential operators, "normal" and partial
%%%%%%%%%%%%%%%%%%%%%%%%%%%%%%%%%%%%%%%%%%%%%%%%%%%%%%%%%%%%%%%%%%%%%%%%%%%

\NewDocumentCommand{\Dn} {m O{} m}  {\frac{\dd^{#1}#2}{\dd{#3}^{#1}}}
\NewDocumentCommand{\Dnp}{m O{} m}  {\frac{\partial^{#1}#2}{\partial{#3}^{#1}}}
\NewDocumentCommand{\Dpp}{m O{} m m}{\frac{\partial^{#1}#2}{\partial{#3}\,\partial{#4}}}
\NewDocumentCommand{\D}  {O{} m}    {\Dn{}[#1]{#2}}
\NewDocumentCommand{\DD} {O{} m}    {\Dn{2}[#1]{#2}}
\NewDocumentCommand{\Dp} {O{} m}    {\Dnp{}[#1]{#2}}
\NewDocumentCommand{\DDp}{O{} m}    {\Dnp{2}[#1]{#2}}

%%%%%%%%%%%%%%%%%%%%%%%%%%%%%%%%%%%%%%%%%%%%%%%%%%%%%%%%%%%%%%%%%%%%%%%%%%%
% Delimiters
%%%%%%%%%%%%%%%%%%%%%%%%%%%%%%%%%%%%%%%%%%%%%%%%%%%%%%%%%%%%%%%%%%%%%%%%%%%

% Define delimiter pairs, resp. appearance of inner product and sets. The final versions are defined below to have default left/right-delimiters and don't crash if both a star and a size argument is present.
% A star is a shortcut to the most important size \big, else an optional argument can be given for the size, see mathtools-documentation. The reason to use these commands is unified appearance, as well as the fact that Latex can't know if vertical bars - i.e. | or \| - are a left or a right delimiter, and thus the spacing would be wrong.
\DeclarePairedDelimiter {\nrmInternal}   {\lVert} {\rVert}
\DeclarePairedDelimiter {\absInternal}   {\lvert} {\rvert}
\DeclarePairedDelimiter {\parInternal}   {\lparen}{\rparen}
\DeclarePairedDelimiter {\braInternal}   {\lbrack}{\rbrack}
\DeclarePairedDelimiter {\ceiInternal}   {\lceil} {\rceil}
\DeclarePairedDelimiter {\flrInternal}   {\lfloor}{\rfloor}
\DeclarePairedDelimiter {\inpInternal}   {\langle}{\rangle}
\DeclarePairedDelimiter {\crlInternal}   {\{}     {\}}
\DeclarePairedDelimiterX{\setInternal}[2]{\{}     {\}}     {#1\;\delimsize|\;#2}

% All of the following commands have the same structure:
% 1. argument: append star to operatorname for \big-delimiters (no star defaults to left/right)
% 2. argument: optional specification of size (\big,\Big,\bigg, etc.)
% 3./4. argument: content between delimiters
\NewDocumentCommand{\norm}   {s o m}  {\resizerOneInput {\nrmInternal}{#1}{#2}{#3}    }
\NewDocumentCommand{\abs}    {s o m}  {\resizerOneInput {\absInternal}{#1}{#2}{#3}    }
\NewDocumentCommand{\snorm}  {s o m}  {\resizerOneInput {\absInternal}{#1}{#2}{#3}    }
\NewDocumentCommand{\card}   {s o m}  {\resizerOneInput {\absInternal}{#1}{#2}{#3}    }
\NewDocumentCommand{\parens} {s o m}  {\resizerOneInput {\parInternal}{#1}{#2}{#3}    }
\NewDocumentCommand{\bracket}{s o m}  {\resizerOneInput {\braInternal}{#1}{#2}{#3}    }
\NewDocumentCommand{\ceil}   {s o m}  {\resizerOneInput {\ceiInternal}{#1}{#2}{#3}    }
\NewDocumentCommand{\floor}  {s o m}  {\resizerOneInput {\flrInternal}{#1}{#2}{#3}    }
\NewDocumentCommand{\inpr}   {s o m}  {\resizerOneInput {\inpInternal}{#1}{#2}{#3}    }
\NewDocumentCommand{\reg}    {s o m}  {\resizerOneInput {\inpInternal}{#1}{#2}{#3}    }
\NewDocumentCommand{\curly}  {s o m}  {\resizerOneInput {\crlInternal}{#1}{#2}{#3}    }
\NewDocumentCommand{\set}    {s o m m}{\resizerTwoInputs{\setInternal}{#1}{#2}{#3}{#4}}

\NewDocumentCommand{\resizerOneInput}{m m m m}{
% First argument is an internal command from \DeclarePairedDelimiter, second is star for \big-delimiters, third is optional size parameter (overridden if star is present!), fourth is content between delimiters.
	\IfBooleanTF{#2}   % star yes/no
		{#1[\big]{#4}} % star triggers \big
		{\IfNoValueTF{#3} % no star -> check if size specified
			{#1*{#4}}     % uses \left...\right
			{#1[#3]{#4}}} % uses size specified
}
\NewDocumentCommand{\resizerTwoInputs}{m m m m m}{
% Like above but with two inputs for internal command from \DeclarePairedDelimiterX
	\IfBooleanTF{#2}
		{#1[\big]{#4}{#5}}
		{\IfNoValueTF{#3}
			{#1*{#4}{#5}}
			{#1[#3]{#4}{#5}}}
}

%%%%%%%%%%%%%%%%%%%%%%%%%%%%%%%%%%%%%%%%%%%%%%%%%%%%%%%%%%%%%%%%%%%%%%%%%%%
% Tools
%%%%%%%%%%%%%%%%%%%%%%%%%%%%%%%%%%%%%%%%%%%%%%%%%%%%%%%%%%%%%%%%%%%%%%%%%%%

\newlength{\hspacetemp} % to be able to use \widthof in hspace, on has to set a length with setlength first, which has to be defined

\newcommand{\lcopywidth}[2]{%
	\phantom{\smash{#2}}\mathllap{#1}}
\newcommand{\ccopywidth}[2]{% 					if used for subscripts, #2 needs to be passed
	\setlength{\hspacetemp}{\widthof{$#2$}}% 	with "\scriptstyle" in front.
	\hspace{0.5\hspacetemp}\mathclap{#1}\vphantom{#2}\hspace{0.5\hspacetemp}}
\newcommand{\rcopywidth}[2]{%
	\mathrlap{#1}\phantom{\smash{#2}}}

\newcommand{\lsetwidth}[2]{%
	\hspace{#2}\mathllap{#1}}
\newcommand{\csetwidth}[2]{%
	\hspace{0.5\dimexpr#2}\mathclap{#1}\hspace{0.5\dimexpr#2}}
\newcommand{\rsetwidth}[2]{%
	\mathrlap{#1}\hspace{#2}}
